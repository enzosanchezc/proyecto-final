\section{Introducción}

% Objetivos específicos. Contenido de los capítulos o secciones 
% Finaliza con un párrafo que resuma el contenido del informe

% En las siguientes secciones se detallarán cada una de estas etapas,
% enumerando los problemas encontrados y las soluciones propuestas.

% Marco teórico
% En esta sección se presenta el desarrollo llevado a cabo para lograr los objetivos propuestos, haciendo énfasis en los aspectos teóricos que fundamentan el proyecto.
%En esta sección se desarrollarán los conceptos teóricos necesarios para entender el funcionamiento del cargador.

% PCB
% A continuación se detallan todos los criterios aplicados para la inclusión de cada componente en el circuito impreso en base a las recomendaciones 
% proporcionadas por la cátedra, los integrantes del ATEI y las especificadas en las hojas de datos de los circuitos integrados.

% Simulaciones del prototipo
% En esta sección se comparan las formas de onda obtenidas con el circuito una vez implementado, con las simulaciones realizadas para el prototipo.

El cargador de baterías de Litio-Ion es un dispositivo electrónico utilizado para suministrar, mediante diferentes etapas, la corriente y tensión continua necesaria para las celdas que componen una batería recargable de forma tal que la misma recupere su carga energética.
Este cargador está diseñado para ser utilizado en baterías de Litio-ion/polímero de 36V. Se alimenta de una tensión de 110 a 240V de corriente alterna, a 50 o 60Hz. Presenta siete modos de carga para baterías de 2.2Ah a 13.2Ah, seleccionables por el usuario, que regulan la intensidad de corriente de salida y su potencia máxima es de 282W.

El proyecto surge de la necesidad de cargar la batería de una bicicleta eléctrica con asistencia al pedaleo,
complementando los proyectos realizados por los alumnos de la cátedra de Proyecto Final durante el año 2021.

El uso de las baterías de litio está en constante crecimiento y,
debido a que es un material altamente reactivo,
es necesario que el proceso de carga se realice de manera correcta,
con la finalidad de ofrecer una carga segura y eficaz.