\section{Recursos,Factibilidad}

Para que exista coherencia entre los objetivos y las actividades que exige el desarrollo del proyecto se requieren diferentes tipos de recursos. 

\subsection{Recursos materiales}
Para la etapa de diseño se contará con bibliografía disponible en Internet o en la biblioteca de la facultad y software de simulación de circuitos.
La construcción del prototipo y los ensayos se realizarán utilizando las herramientas, el equipamiento y la instrumentación del 
Área Técnica de Electrónica e Instrumental (ATEI) del Departamento de Electrotecnia de la Facultad de Ingeniería de la UNLP. 
Para la validación experimental del prototipo se requerirá una batería de 36V con BMS integrado. 

\subsection{Recursos financieros}
Los fondos necesarios para la compra de componentes serán proporcionados por la cátedra de Proyecto Final.

\subsection{Recursos humanos}
Las actividades serán llevadas a cabo por el equipo que conforma el proyecto con el apoyo de los integrantes de la cátedra.

Dado que el proyecto no depende de factores socioculturales o políticos, en base a la disponibilidad y correcta asignación
de todos los recursos previamente nombrados, es factible y viable cumplir con los objetivos y las metas establecidas. 