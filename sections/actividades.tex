\section*{Actividades y Metodología}
El proyecto se llevará a cabo en la Facultad de Ingeniería de la Universidad Nacional de La Plata.
Los ensayos necesarios se realizarán en el ATEI (Área Técnica de Electrónica e Instrumental),
ubicado dentro del edificio de Electrotecnia en la facultad.
Se realizarán consultas semanales con el tutor para el seguimiento del desarrollo del proyecto y revisión de decisiones.

El diseño será dividido en distintos bloques, cada uno de los cuales se detallará en el siguiente listado:
\begin{itemize}
    \item Fuente conmutada: Realiza la conversión de tensión de la red domestica a una tensión continua para alimentar el cargador.
    \item Fuente de corriente: Brinda una corriente constante a la batería durante la primera etapa de carga.
    \item Circuito de control: Se encarga de alternar entre las etapas de carga.
    \item Circuito de protección: Debe proteger a la batería en caso de cortocircuito.
\end{itemize} 

Lectura de bibliografía. Buscar, analizar y seleccionar información en diversas fuentes.

Desarrollo de diferentes propuestas y toma de decisiones en base a las especificaciones recomendaciones del tutor

Cálculo de los componentes necesarios en base a la topología utilizada

Elaboración de documento con la información utilizada para su diseño 

Simulación del diseño del circuito mediante software

Creación de un prototipo a escala

Pruebas técnicas que permitan evaluar el grado del cumplimiento del proyecto

estrategias y técnicas