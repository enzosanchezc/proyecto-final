\section{Actividades y Metodología}

%Localización física y ubicación de instalaciones:

El proyecto se llevará a cabo en la Facultad de Ingeniería de la Universidad Nacional de La Plata.
Los ensayos necesarios se realizarán en el ATEI (Área Técnica de Electrónica e Instrumental),
ubicado dentro del edificio de Electrotecnia en la facultad.
Se realizarán consultas semanales con el tutor para el seguimiento del desarrollo del proyecto y
revisión de decisiones tomadas por el equipo de trabajo.

Para alcanzar las metas y los objetivos propuestos, se realizarán las siguientes actividades integradas y secuenciales:

\subsection{Lectura de bibliografía}
Con el objetivo de capacitarse, se estudiarán y analizarán aspectos de seguridad, curvas de carga de la batería 
y topologías de conversores de potencia, de acuerdo con los requerimientos del proyecto. 
Se evaluarán las diferentes alternativas posibles y,
en base a la complejidad y al costo de las diferentes alternativas,
se elegirá la solución más adecuada para el logro de los objetivos. 

\subsection{Diseño}
Una vez estudiados los aspectos mencionados en el apartado anterior,
se comenzará con el diseño del cargador,
realizando simulaciones en SPICE (Programa de simulación con énfasis en circuitos integrados),
separando el proceso en 4 partes:
\begin{itemize}
    \item Fuente conmutada: Realiza la conversión de tensión de la red domestica a una tensión continua para alimentar el cargador.
    \item Fuente de corriente: Brinda una corriente constante a la batería durante la primera etapa de carga.
    \item Circuito de control: Se encarga de alternar entre las etapas de carga.
    \item Circuito de protección: Debe proteger a la batería en caso de cortocircuito.
\end{itemize}

\subsection{Implementación}
Se realizará una implementación del cargador en una placa de circuito impreso.
Debido a que el coste de una batería de litio de las características necesarias es elevado,
no se descarta la posibilidad de realizar un prototipo a escalado a una batería de menor tensión y/o capacidad.

\subsection{Validación}
Una vez terminado el proceso de diseño e implementación,
se procederá a verificar que el prototipo cumpla con los requerimientos del proyecto.

%Desarrollo de diferentes propuestas y toma de decisiones en base a las especificaciones recomendaciones del tutor

%Cálculo de los componentes necesarios en base a la topología utilizada

%Elaboración de documento con la información utilizada para su diseño 

%Simulación del diseño del circuito mediante software

%Creación de un prototipo a escala

%Pruebas técnicas que permitan evaluar el grado del cumplimiento del proyecto

%estrategias y técnicas