\section{Actividades y Metodología}

%Localización física y ubicación de instalaciones:

El proyecto se llevó a cabo en la Facultad de Ingeniería de la Universidad Nacional de La Plata.
Los ensayos necesarios se realizaron en el Área Técnica de Electrónica e Instrumental (ATEI)
del Departamento de Electrotecnia de la Facultad de Ingeniería de la UNLP.
Se realizaron consultas semanales con el tutor para el seguimiento del desarrollo del proyecto y
revisión de las decisiones tomadas por el equipo de trabajo.

Para alcanzar las metas y los objetivos propuestos, se llevaron a cabo las siguientes actividades:

\subsection{Estudio de bibliografía y diseño} \label{subsection:estudio_bibliografia}
Con el objetivo de capacitarse, se estudiaron y analizaron aspectos de seguridad, curvas de carga de la batería 
y topologías de conversores de potencia, de acuerdo con los requerimientos del proyecto. 
Se evaluaron las diferentes alternativas posibles y, en base a su complejidad y a su costo,
se eligió la solución más adecuada para el logro de los objetivos. 

Se realizaron simulaciones en SPICE (programa de simulación con énfasis en circuitos integrados),
separando el proceso en 4 partes:
\begin{itemize}
    \item Fuente conmutada: Convierte la tensión alterna de la red doméstica en una tensión continua.
    \item Fuente de corriente: Brinda una corriente constante a la batería durante la primera etapa de carga.
    \item Circuito de control: Alterna entre las etapas de carga.
    \item Circuito de protección: Protege a la batería en caso de cortocircuito.
\end{itemize}

En la Figura \ref{fig:esquema_cargador} se puede observar un esquema en bloques del cargador.
El circuito de la fuente conmutada está compuesto por los bloques de rectificación, filtrado y conversión DC-DC.
El controlador se encarga de generar una señal PWM en base a la tensión y la corriente de la batería.
La referencia es una señal de corriente ya que la tensión nominal del cargador es fija.
El circuito de protección está integrado en el bloque de control.

Para disminuir las pérdidas de potencia en la etapa de corriente constante y obtener una mayor eficiencia, se modificó la estructura del cargador con respecto al diseño original. 
En primera instancia, para mantener la tensión a la salida del conversor en 42V, se propuso un ciclo de trabajo constante 
con el cual la caída de tensión en la etapa de control de corriente generaba una disipación de potencia excesiva.
Modificando el circuito de conversión que controla tanto la corriente como la tensión se evita una etapa posterior limitadora de corriente.

%incluir esquema de cargador
\begin{figure}
    \centering
    \includegraphics[width=\textwidth]{images/esquema_cargador_v2.png}
    \caption{Esquema del cargador}
    \label{fig:esquema_cargador}
\end{figure}

Se realizó un análisis crítico de las primeras simulaciones y, en base a los resultados obtenidos, 
se corrigieron los circuitos propuestos. 
Esta etapa finaliza con la presentación del informe parcial. 

\subsection{Simulaciones}
El circuito fue diseñado y probado en LTspice \cite{ltspice}, con el fin de verificar el diseño.
El proceso se dividió en las siguientes etapas:
\begin{enumerate}
    \item Simulación del conversor.
    \item Simulación del rectificador de entrada.
    \item Simulación del circuito de control.
    \item Simulación de distintos modelos de batería.
    \item Simulación del driver para el MOSFET high-side.
\end{enumerate}

En las siguientes secciones se detallarán cada una de estas etapas,
enumerando los problemas encontrados y las soluciones propuestas.

\subsection{Implementación y validación}
Se construyó un prototipo del cargador, realizando una primera aproximación con una placa perforada
y luego diseñando un circuito impreso para la implementación final.

Por último, se realizaron mediciones de tensión y corriente en el circuito implementado,
con el fin de validar los resultados obtenidos en las simulaciones.
