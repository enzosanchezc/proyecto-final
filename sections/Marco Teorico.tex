\section{Marco teórico}

Para reducir el tamaño del transformador y cumplir con las especificaciones normalmente se requieren conversiones de etapas múltiples.
Las fuentes de potencia otorgan una alta densidad de potencia en un tamaño y con un peso reducido.  
Permiten aislar eléctricamente a la carga de la red de alimentación. 
Dirección controlada del flujo de potencia. 
Alta eficiencia de conversión. 
Utilizando pequeños filtros es posible tener formas de onda con baja distorsión armónica tanto en la entrada como en la salida. 
Permiten controlar el factor de potencia si la fuente es de AC. 

En base a la tensión de salida requerida existen fuentes de alimentación AC y fuentes de alimentación DC.
Las fuentes de alimentación DC se clasifican en:
1) Conmutación
Tienen una alta eficiencia y pueden suministrar altas corrientes de carga a una tensión baja.
Existen 5 topologías comunes: fly-back, forward, push–pull, half-bridge, y full-bridge.
Por lo general se utilizan 2 etapas de conversión: DC-AC mediante modulación de ancho de pulso (PWM) de la etapa del inversor y de AC-DC.
La salida del inversor, que varía mediante una técnica PWM, se convierte en un voltaje de DC mediante un rectificador de diodos. 
Debido a que el inversor puede operar a una frecuencia muy alta, las fluctuaciones en la tensión de salida de DC se pueden filtrar fácilmente con filtros pequeños. 
2) Resonantes
Si la variación del voltaje de salida no es amplia se pueden usar inversores de pulso resonante. 
La frecuencia del inversor, que podría ser la misma que la frecuencia de resonancia, es muy alta y la tensión de salida del inversor es casi sinusoidal 
Debido a la oscilación resonante, el núcleo del transformador siempre se restablece y no hay problemas de saturación. 
Los tamaños del transformador y del filtro de salida se reducen debido a la alta frecuencia del inversor.
3) Bidireccionales 
Aptas para carga y descarga de baterías donde el flujo de potencia es bidireccional. 
El flujo de potencia depende de la tensión de entrada, de la tensión de salida y de la relación de vueltas del transformador. 
Permiten que la corriente inductiva fluya en cualquier dirección y que el flujo de corriente se vuelve continuo.
Requiere sintetizar las funciones de conmutación para obtener las formas de onda de salida deseadas.

Especificaciones de la tensión de salida del prototipo: 12.6VDC
Para seleccionar una topología adecuada para una aplicación, es necesario comprender las ventajas y desventajas de cada topología y los requisitos de la aplicación. 
Existen diversas topologías, las cuales serán analizadas en orden creciente de complejidad. 
Aunque la mayoría de los convertidores se pueden utilizar para cumplir con los requerimientos de salida, 
los valores nominales del dispositivo de conmutación y el tamaño del transformador limitan sus aplicaciones a una potencia de salida específica. 
La elección del convertidor depende del requisito de potencia de salida y de la complejidad que se desea afrontar.


Conversor Forward


El convertidor forward es un convertidor CC-CC acoplado magnéticamente. 

El transistor funciona como interruptor, estará cerrado un tiempo DT y abierto el resto del tiempo,
 (1 - D)T, siendo T el periodo de conmutación. 
Para el análisis del circuito se supondrá funcionamiento 
en régimen permanente y que la corriente en la inductancia Lx es permanente.
El transformador posee tres devanados: los devanados 1 y 2 transfieren la energía de la
fuente a la carga cuando el interruptor está cerrado; el devanado 3 se usa para proporcionar un
camino a la corriente magnetizante cuando el interruptor está abierto y reducirla a cero antes del
inicio de cada periodo de conmutación. El transformador se modela como tres devanados ideales
con una inductancia magnetizante Lm conectada en paralelo con el devanado 1. En este modelo
de transformador simplificado, no se incluyen las pérdidas ni las inductancias de dispersión.
En el convertidor forward, la energía del generador se transfiere a la carga cuando el interruptor
está cerrado. En el convertidor flyback, la energía se almacenaba en Lm cuando el conmutador
estaba cerrado y era transferida a la carga cuando estaba abierto. En el convertidor directo, Lm
es un parámetro no incluido en la relación entrada-salida, y se suele hacer grande su valor

Análisis con el interruptor cerrado
En la Figura 7.5b se muestra el circuito equivalente del convertidor forward cuando el interruptor
está cerrado. Al cerrarse el interruptor se establece una tensión en el devanado 1 del transformador,
por lo que

La corriente en Lm deberá anularse antes del inicio del siguiente período, para desmagnetizar
el núcleo del transformador. Cuando se abre el interruptor, la Ecuación 7.26 indica que la corriente
iL decrece linealmente. Como D3 impide que la corriente iL se haga negativa, la Ecuación
7.26m será válida siempre que iL sea positiva. Utilizando la Ecuación 7.26 obtenemos



Para que la corriente iL se anule una vez abierto el interruptor, la disminución de corriente
debe ser igual al incremento de la corriente indicado en la Ecuación 7 .20. Si el tiempo necesario
para que la corriente iL de pico se anule es deltaTx, 


Resolviendo para obtener deltaTx, 


El instante t0 en el que se anula la corriente es


Teniendo en cuenta que la corriente debe anularse antes del inicio del siguiente periodo,


Por ejemplo, si la relación N3/N1 = 1, el ciclo de trabajo D deberá ser menor que 0,5. La tensión
en el interruptor abierto es Vs-v1, por lo que


La configuración del circuito a la salida del convertidor forward es la misma que la del convertidor
reductor, por lo que el rizado de la tensión de salida también será el mismo:


Cuando el interruptor está cerrado, la fuente entrega energía a la carga a través del transformador.
La tensión en el secundario del transformador es una forma de onda pulsante y la salida se
analiza de la misma manera que la del convertidor CC-CC reductor. La energía almacenada en
274 Electrónica de potencia
la inductancia magnetizante cuando el interruptor está cerrado puede ser devuelta a la fuente de
entrada a través de un tercer devanado del transformador cuando el interruptor está abierto.

Restablecimiento del núcleo del transformador:
La energía almacenada en el núcleo del transformador es devuelta a la fuente regulable y se incrementa la eficiencia. 

A diferencia del convertidor flyback, la energía no se almacena en el primario y sólo se opera 
en el modo de conducción continua por la mayor dificultad del control en base al doble polo existente en el filtro de salida. 
Asumiendo modo de conducción continua, operación en estado estacionario y ripple de salida nulo, 
existen 2 modos de operación del transistor:

Cuando el transistor se encuentre encendido:

La tensión en el bobinado primario es Vs. La corriente que circula por el mismo comienza a incrementarse 
y se transfiere energía del primario al secundario y de aquí al filtro de salida y la carga por medio del diodo D2. 
Debido a esta corriente se induce una corriente en el secundario dada por:

La corriente magnetizante se incrementa linealmente con el tiempo:

La corriente total que circula por el primario resulta:

Cuando finaliza el tiempo de conducción del transistor en un tiempo t=DT, esta corriente total llega a un valor máximo dado por:

donde XXX es la corriente pico reflejada del inductor de salida del secundario y está dada por:

La tensión inducida en el secundario es:

Debido a la tensión Vsecundario-Vo existente en bornes del inductor, su corriente se incrementa linealmente:

Esta misma también tendrá su valor máximo en t=DT:


Cuando el transistor se encuentre apagado:

La tensión del transformador toma polaridad negativa, lo cual apaga al diodo D2 y enciende a los diodos D1 y D3. 
Mientras conduce D3, la energía es entregada a la carga a través del inductor de salida. 
La corriente por ambos dispositivos es la misma y decrece linealmente con el tiempo:

Lo cual nos da el valor de I1(0):


El diodo D1 y el tercer bobinado del transformador le proporcionan un camino 
a la corriente magnetizante para que regrese a la fuente de entrada. 
La tensión de salida es la integral en el tiempo de la tensión del bobinado secundario:

La máxima corriente de colector se da durante el encendido del transistor y la máxima tensión de colector durante el apagado:


Igualando la integral en el tiempo de la tensión de entrada cuando el transistor está encendido a la tensión Vr cuando el transistor está apagado nos da el ciclo de trabajo máximo:


El ciclo de trabajo debe mantenerse siempre debajo del máximo para evitar la saturación del núcleo del transformador. 
Además, la corriente debe ser reseteada a 0 al final de cada ciclo de conmutación. 
Si el núcleo está saturado, se puede dañar el transistor. 

Comparación con el conversor flyback:

El conversor forward requiere de una carga mínima para evitar un exceso en la tensión de salida. 
Como el transformador no almacena energía, para un mismo nivel de potencia de salida, 
el tamaño del mismo es menor en el convertidor forward que en el flyback. 
La corriente de salida es aproximadamente constante ya que el ripple disminuye notablemente 
debido al agregado del inductor en la salida y al diodo de rueda libre D3.
Por esto mismo, el capacitor de salida puede ser más pequeño. 


Conversor Forward Doble Switch

El conversor forward se utiliza para potencias de salida de hasta 200W ya que se encuentra limitado 
por los esfuerzos de tensión y de corriente durante su funcionamiento. 
El conversor forward de doble switch permite ser utilizado con potencias de hasta X W debido a la 
reducción de la tensión en los transistores cuando los mismos se encuentran apagados. 
Esto permite su uso en aplicaciones de alta tensión mediante transistores de menores prestaciones en tensión. 

Funcionamiento 

Los transistores se encienden y se apagan de forma simultánea. 
Cuando los transistores están encendidos, la tensión en el primario del transformador es igual a la tensión de entrada Vs. 
Como la relación del mismo es 1:1, la misma tensión se establece en su secundario y la energía se transfiere a la carga. 
Además se incrementa la corriente que circula por la inductancia magnetizante. 

Cuando los transistores se apagan, el diodo D1 evita que la corriente magnetizante circule por el secundario 
(y por lo tanto también en el primario) del transformador, forzando su camino por los diodos D3 y D4 de regreso a la fuente regulable.  
Con esto se elimina la necesidad del tercer devanado de desmagnetización. 
La tensión en el primario del transformador es -Vs, causando un decremento de la corriente magnetizante. 
Si la relación de trabajo de los transistores es menor a 0.5, en cada ciclo el núcleo del transformador se restablece, 
haciendo que el flujo magnético regrese a 0. 
La tensión en los transistores cuando los mismos se encuentran apagados es Vs y no Vs(1+N1/N3).

La tensión de salida resulta:
$$ V_{o}=V_{s}D(\frac{N_{2}}{N_{1}}) $$




Circuitos de control

La tensión de salida del convertidor puede ser controlada variando el ciclo de trabajo D. 
Para ello se utilizan circuitos integrados controladores PWM que sólo requieren de unos pocos componentes pasivos adicionales para su funcionamiento. 
Internamente presenta 4 componentes principales:
1) Un reloj ajustable que permite configurar la frecuencia de conmutación 
2) Amplificador de error para la tensión de salida
3) Generador de forma de onda de dientes de sierra sincronizado con el reloj
4) Un comparador para comparar la señal de salida de error con la señal de dientes de sierra.
Su señal de salida es la que controla a los transistores. 

Para controlar la tensión de salida, los convertidores funcionan con un circuito de retroalimentación,
 en base a la señal realimentada , en un control por tensión o corriente.

Control por tensión

COMPLETAR CON LIBRO

La duración del tiempo de encendido está determinada por el tiempo entre el reinicio del generador de diente de sierra
 y la intersección del voltaje de error con la señal de rampa positiva. 
 Cuando la tensión de salida es inferior al valor nominal se genera una tensión de error. 
 El ciclo de trabajo aumenta para causar un aumento posterior en el voltaje de salida. 
 La dinámica de retroalimentación está determinada por el circuito amplificador de error que consta de Z1 y Z2.



 Control por corriente 

 Consiste en un lazo interno que muestrea el valor de la corriente primaria y apaga los interruptores tan pronto como la corriente alcanza cierto valor establecido por el lazo de voltaje externo. 
 De esta manera, el control de corriente logra una respuesta más rápida que el modo de voltaje. 
 La forma de onda de corriente primaria actúa como onda de diente de sierra. 
 El voltaje análogo a la corriente puede ser proporcionado por una pequeña resistencia o por un transformador de corriente. 
 La figura 13.19a muestra un convertidor flyback controlado por modo de corriente, donde la corriente del interruptor isw se usa como señal portadora. 
 La corriente del interruptor isw produce un voltaje a través de Rs, que se retroalimenta al comparador. 
 El encendido está sincronizado con el pulso del reloj y el apagado está determinado por el instante en que la corriente de entrada es igual al voltaje de error.
 Debido a su capacidad inherente de limitación de corriente máxima, el control de modo de corriente puede mejorar la confiabilidad de los interruptores de alimentación. El rendimiento dinámico se mejora debido al uso de la información actual adicional. El control de modo de corriente reduce efectivamente el sistema a primer orden al obligar a que la corriente del inductor se relacione con el voltaje de salida, logrando así una respuesta más rápida. Las figuras 13.18b–e muestran las formas de onda.

 COMPLETAR CON JUSTIFICACIÓN DE POR QUÉ NO ELEGIMOS CONTROL POR CORRIENTE QUE PARECERÍA SER MEJOR. 

 TL494 

El TL494 es un circuito de control de modulación de ancho de pulso (PWM) de frecuencia fija. 
La modulación de los pulsos de salida se logra comparando la forma de onda de diente de sierra creada por el oscilador interno en el capacitor de temporización (CT) con cualquiera de las dos señales de control. 
La etapa de salida se habilita durante el tiempo en que el voltaje de diente de sierra es mayor que las señales de control de voltaje. 
A medida que aumenta la señal de control, disminuye el tiempo durante el cual la entrada en diente de sierra es mayor; por lo tanto, la duración del pulso de salida disminuye. 
Un flip-flop de dirección de pulso dirige alternativamente el pulso modulado a cada uno de los dos transistores de salida.


Alimentación: 7-40V.
El TL494 está diseñado para operar desde un rango de suministro de voltaje de entrada entre 7 V y 40 V.

Feedback: sin conexión ya que se trabaja a lazo abierto

Regulador interno de 5V con precisión del 5\%. 

Oscilador interno ajustable 
Provee la forma de onda de diente de sierra al tiempo muerto y al comparador PWM. 
Su frecuencia se programa mediante la selección de una resistencia Rt y un capacitor Ct. 
El oscilador carga al Ct con una corriente constante determinada por Rt produciendo una rampa de tensión sobre Ct.
Icarga=3V/Rt.
Cuando la tensión sobre el mismo llega a 3V, el circuito se descarga y se reinicia el ciclo. 
f=1/(Rt*Ct) para single ended 
% COMPLETAR CON COMPONENTES ELEGIDOS PARA f=125KHz

Control de tiempo muerto (DTC)
Permite controlar el ciclo de trabajo. 
Es una entrada de alta impedancia. 
Controla el tiempo de apagado mínimo. Con DTC a tierra es del 3%.
Si se aplica tensión en este puerto se le puede adicionar.
El tiempo muerto o de apagado se controla linealmente desde su mínimo de 3\% hasta su máximo de 100\%, 
variando su tensión de entrada entre 0 y 3.3V respectivamente. 

Comparador 
Alimentado por el regulador interno. 
Tiempo de respuesta de 400ns. 
Modula el ancho de pulso de la salida. Para esto se compara la rampa de tensión sobre el capacitor Ct con la señal de control
presente en la salida de los amplificadores de error. 

2 amplificadores de error 
Alta ganancia
Se alimentan mediante su entrada Vi. 
Tensión de MC de entrada: -0.3V a Vcc-2V

Output-Control Input
Determinan el modo de operación de la salida de los transistores. 
Si está a tierra opera en modo single-ended o modo paralelo donde los pulsos vistos en la salida del DTC son transmitidos por ambos transistores de salida en paralelo.
Si está a Vref opera en push-pull donde cada transistor de salida está habilitado alternativamente por el flip-flop de dirección de pulsos.

Transistores de salida
El integrado incluye 2 transistores que tienen la posibilidad de ser emisor común o seguidor por emisor. 
Son capaces de generar hasta 200mA de salida. 


Transformador: permite que no circule corriente continua por su secundario. 

Amplificadores de potencia

Amplificador Clase B con transistores complementarios

Permite acoplar la carga en continua. 
El dispositivo de amplificación conduce durante medio ciclo de la señal de entrada. 
El transistor sólo puede reproducir medio ciclo de la señal de entrada. 
Se excita al transistor inyectándole corriente por su base.
Los transistores son de potencia y no de señal. 
Se compone de un transistor NPN y otro PNP que se alimentan de forma inversa.
El transistor NPN requiere de una tensión positiva en su colector y 
el transistor PNP requiere de una tensión negativa en su colector.
Ambos funcionan como seguidor por emisor o colector común, con su tensión de alimentación en colector, la señal de excitación en la base y la carga conectada el emisor. 
Su ganancia de tensión es aproximadamente 1. 
La topología seguidor por emisor tiene ganancia de corriente hfe. 
La corriente que se entrega por la base es hfe veces menor que la que se le entrega a la carga. 
Los transistores conducen en base a la tensión Vbe en sus junturas ya que entra base y emisor existe una juntura PN similar a un diodo. 
Si no se polariza de forma correcta, no circula corriente por su base y en consecuencia tampoco por su colector. 
El transistor NPN conduce corriente en su emisor cuando Vbe>0. 
El transistor PNP conduce corriente en su emisor cuando Vbe<0 o Veb>0.
En el semiciclo positivo de la señal conduce el NPN y el PNP se corta y en el semiciclo negativo conduce el PNP y el NPN se corta. 
Los transistores conducen de forma alternativa. 
Cuando la tensión de entrada en la base de ambos transistores es 0 no conduce ningún transistor. 
Esto implica que no existe consumo de corriente ni de potencia. 
Vce de ambos transistores va de Vcc a 0.

Seguidor

Su impedancia de entrada es mucho mas alta que su impedancia de salida, de modo que una fuente de señal no tendría que trabajar tan duro.
 Esto puede verse en el hecho de que la corriente de base es del orden de 100 veces menos que la corriente de emisor. 
 La baja impedancia de salida del seguidor emisor se adapta con una carga de baja impedancia y amortigua la fuente de señal.