\section{Conclusiones}

% EMPEZAR CON LAS COSAS BUENAS: Se logró, se implementó, ...
% Recomendaciones 

La tensión de salida no se mantiene estable ante variaciones en la carga con ciclo de trabajo y tensión de alimentación fija. 
Esto puede deberse a que se consideró el modelo ideal de todos los componentes del circuito al momento de realizar el análisis del conversor. 

En la teoría analizada por la bibliografía se supone que todos los componentes del convertidor forward son ideales. 
Esto se evidencia en el simulador donde al utilizar modelos reales de los semiconductores se comienza a evidenciar como la tensión de salida varía levemente con la carga. 
Esta suposición realizada para simplificar el análisis lleva a tener diferencias en la práctica principalmente con el comportamiento de la tensión de salida. 
Ejemplos: resistencia del inductor del filtro de salida, pérdidas en los MOSFETs,
Se evidencia como ante el aumento de la corriente de carga la tensión de salida obtenida disminuye y se aleja del comportamiento semi-ideal del simulador. 
Como las pérdidas en todos los componentes aumentan con la corriente de carga. 
Es por ello que en la práctica los convertidores funcionan a lazo cerrado sensando la tensión o corriente de salida, y por medio de un sistema de realimentación ajustan el ciclo de trabajo para obtener una tensión de salida constante ante los cambios de carga o la tensión de entrada del convertidor.

% Completar

Quedó pendiente por implementar el rectificador a la entrada del conversor y el circuito de control.