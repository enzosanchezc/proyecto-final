\section{Problemas afrontados y soluciones}

TL494

Una incorrecta elección de Rt y Ct fuera de los valores recomendados generaba variaciones muy grandes de la frecuencia de switching, 
lo cual causaba inestabilidad en la tensión de salida. 

MEDIDOR DE CAPACITANCIA E INDUCTANCIA DM6243L

Con este instrumento se obtuvo el valor de inductancia del inductor de la salida y de la inductancia de magnetización del transformador. 
Para el transformador, con los extremos de un bobinado a circuito abierto, se realizó la conexión del instrumento al otro bobinado y se obtuvo:
Transformador de señal E25:
Lm=0.186mH; 
Transformador E70
Lm=73uH; AMARILLO 
Lm=84uH; NEGRO
Inductor de salida:
L=340uH;

Precisión: +-(2\% del FE+Número de dígitos)
FE=2mH
Número de dígitos=5
COMPLETAR CÁLCULO 

OSCILACIONES

Las oscilaciones de alta frecuencia (superior a 125kHz) observadas en las formas de onda se pueden deber a cualquier tanque LC. 

CABLES

Para reducir el ruido se utilizó la menor longitud posible para los cables utilizados en las conexiones. 

ESQUEMÁTICO GENERAL 

Sacar de EasyEda. 

IDENTIFICACIÓN DE PUNTOS HOMÓLOGOS UNA VEZ REALIZADOS 

Texto.

IMPLEMENTACIÓN

Se agregaron puntos de prueba en el circuito para realizar todas las mediciones necesarias. 

CONSUMO DE CORRIENTE DE COMPONENTES 

En base a la corriente medida en las resistencias de bajo valor y el display de las fuentes de tensión utilizadas. 

CÁLCULO DEL CICLO DE TRABAJO UTILIZADO 

D=TonPWM/Tswitching=125kHz*TonPWM

Destacar la estabilidad de la frecuencia de conmutación y la variación de la amplitud de la forma de onda PWM con la variación del ciclo de trabajo. 

POSIBILIDAD DE UTILIZAR DISIPADOR PARA MOSFET

PRIMERAS PRUEBAS

Al conectar la señal de salida PWM del TL494 al convertidor forward, debido a la carga vista, su amplitud disminuye de 12V a menos de 0.1V. 
Actualmente con la etapa de ganancia de corriente, la señal PWM disminuye a una amplitud de XV y presenta una leve deformación.

¿BUFFERS YA NO ESTÁN MÁS?
Prueba con LM324N
Producto Ganancia por Ancho de Banda: 1.3MHz y Slew Rate: 0.4V/us. 
Alimentación 12V. Los amplificadores operacionales que componen al circuito integrado no son rápidos ya que no pueden seguir a la tensión en su entrada. 
ADJUNTAR IMÁGEN!
Presenta tiempos de subida y de bajada muy lentos. En base a la medición realizada se necesita aproximadamente 1V/us de slew rate. 
Dado que el problema es el slew rate se busca un comparador en lugar de un amplificador operacional. 
DUDOSO:
La principal diferencia entre un amplificador operacional y un comparador es que el comparador
 no tiene el capacitor en su estructura interna que es el que limita el slew rate en el amplificador operacional. 

COMPARADOR LM339

Rango de alimentación: 2-36V
COMPLETAR
Velocidad: 1.3us

Se realizaron pruebas con un comparador. 
Salida por colector. Corriente de salida: 6-16mA. Necesita una resistencia de pull-up que fije esta corriente de $2k\Omega$ y $750\Omega$ respectivamente. 
Las señales de entrada eran la propia señal PWM que se desea mejorar y una referencia cuya amplitud 
sea la mitad de la primer señal implementada mediante un divisor de tensión. 
Adicional: Realimentación positiva con schmitt trigger para las transiciones. ¿Se implementó?+

NO ES NECESARIO LA ETAPA DE GANANCIA 2 Y BUFFER ¿Por qué ya no es necesaria?

Dado que la corriente máxima por colector del TL494 es 250mA, el integrado puede comandar directamente al driver del MOSFET high side. 

DRIVER

Los capacitores de desacople actúan como un cortocircuito en la frecuencia de operación. 
Diodo evita la parte negativa. 
Analizar la Vg para no superar la máxima. 
El transformador sin componentes adicionales consumía demasiada corriente porque se lo alimentaba con corriente continua. 

TRANSFORMADOR DE POTENCIA

Núcleo: E70
Con el instrumento:
Primario(cables amarillos): Lm=81uH
Secundario(cables negros): Lm=76uH

INDUCTOR

Núcleo: E42
Número de vueltas: N=36
Gap: 0.2mm

Con un circuito RL serie y un generador de funciones se mide la inductancia en la frecuencia de interés.
Para una resistencia de $R=200\Omega$ y una onda rectangular como excitación de 125kHz de 1.5V de amplitud, se obtiene:
tao=2.34us=L/R
L=R*tao=$200\Omega$*2.34us=468uH

Con el instrumento se mide 323uH. El mismo no aclara la frecuencia de medición en su hoja de datos. 
L=Al*N^2 => Al=249.22nH
Este valor de Al no se corresponde con el gap introducido ya que el núcleo presenta un gap superior en su centro. 

La medición a partir de Vl=L*dil/dt, sobre el propio convertidor midiendo la tensión aplicada sobre el inductor y la pendiente de la corriente, 
no se pudo realizar en la placa de pruebas ya que las formas de onda están deformadas. 
La resistencia serie que se puso para medir la corriente fue de 12Ohms. 

TRANSFORMADOR DE SEÑAL 

Núcleo: E30
Con el instrumento:
Primario: Lm=187uH
Secundario: Lm=186uH
Respecto a la hoja de datos, tiene el doble de espesor. 
Por lo tanto, el área efectiva se duplica. 

RECOMENDACIONES

Se podría haber implementado el transformador del driver con un núcleo toroidal. 

INDUCTANCIA MAGNETIZANTE

Medir Lm mediante constante de tiempo de la tensión del capacitor de acoplamiento del driver.

0\% -240mV
100\% 760mV
DeltaV=1V
10\% -140mV
90\% 660mV
Con esto:
Tao=150ns. 

DIODO SCHOTTKY

Queda pendiente la prueba. 

CAPACITOR DE SALIDA 

Medir ESR del capacitor utilizado
No se hizo. 

RECALCULAR INDUCTOR CON MATLAB

CHEQUEAR EFICIENCIA DE LA META ESTABLECIDA 

15/11

Problemas:

1) La fuente de tensión de potencia eleva notablemente su corriente entregada al circuito 
cuando se conecta la tierra de la punta del osciloscopio en ciertos puntos de la PCB. 
Además disminuye la tensión entregada por la fuente de 12V. 
Ejemplo: arriba de la Rgs.

2) Al utilizar resistencias de $1\Omega$ y potencia nominal 2Watts para medir corrientes en el circuito, 
se detectaron oscilaciones con una frecuencia de 1.1MHz. Ejemplo: Id1 cuando debería ser constante, 
Id2 cuando debería ser constante y en la rampa. 

Solución: 

Se quitaron las resistencias y se soldaron cables con un largo tal que se pueda medir la corriente a través de los mismos 
utilizando unas puntas de corriente para el osciloscopio. Al realizar las mismas mediciones no se detectaron las oscilaciones,
con lo cual se concluye que a la frecuencia de trabajo las resistencias dejan de comportarse como tales y presentan una alta componente inductiva. 

Condiciones experimentales:

Ciclo de trabajo: D=0.4 (t=3.2us/8us)
Resistencia de carga: $138\Omega$

Estado estacionario: 
Fuente 1: 11.9V-0.05A
Fuente 2: 36V-0.12A
Carga: 7.67V-0.05A

16/11

Problema con las tierras.

Puntas TEKTRONIX modelo P6021 con 2mA/mV o 10mA/mV. Se deben bloquear para realizar las mediciones.
Están conformadas por un transformador de corriente. 
El primario consta de una única vuelta ya que es el cable que se introduce y el secundario está adentro de la misma. 
Las formas de onda que presentan continua deben ajustarse teniendo en cuenta que las mismas sólo miden alterna. 

Capacitor electrolítico de 200uF se reemplaza por uno de tantalio de 10uF. Los mismos presentan bajo ESR. 

Al quitar las resistencias se incrementa levemente la tensión de salida y la eficiencia. 

En ocasiones tiene lugar un transitorio con mayor demanda de corriente de la fuente de potencia. 

Existen oscilaciones en muchas partes del circuito:

Tensión en el inductor: 1.72MHz
Primario del E70: 1.32MHz
Secundario del E70: 3.14MHz

17/11

Con el transformador 1:2, 
Lamarillo=0.066mH
Lnegro=0.293mH

Número de vueltas del primario: 
Número de vueltas del secundario: N2=40 

Tabla:
 
%INVENTAR VALORES QUE FALTAN

R[\Omega]       Ii[A]       Vo[V]       Io[A]
12.6            0.23        5.77        0.47
18              0.25        7           0.4
24.3            0.23        8.3         0.35
30.6            0.23        9.13        0.3
40                          10.2        0.25
58                          11.6        0.2
80.7            0.16        12.4        0.15           
88.5            0.16        12.6        0.14     
165             0.16        14.27       0.09   