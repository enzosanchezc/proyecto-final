\section{Problemas afrontados y soluciones}

TL494

Una incorrecta elección de Rt y Ct fuera de los valores recomendados generaba variaciones muy grandes de la frecuencia de switching, 
lo cual causaba inestabilidad en la tensión de salida. 

MEDIDOR DE CAPACITANCIA E INDUCTANCIA DM6243L

Con este instrumento se obtuvo el valor de inductancia del inductor de la salida y de la inductancia de magnetización del transformador. 
Para el transformador, con los extremos de un bobinado a circuito abierto, se realizó la conexión del instrumento al otro bobinado y se obtuvo:
Transformador de señal E25:
Lm=0.186mH; 
Transformador E70
Lm=73uH; AMARILLO 
Lm=84uH; NEGRO
Inductor de salida:
L=340uH;

OSCILACIONES

Las oscilaciones de alta frecuencia (superior a 125kHz) observadas en las formas de onda se pueden deber a cualquier tanque LC. 

CABLES

Para reducir el ruido se utilizó la menor longitud posible para los cables utilizados en las conexiones. 

ESQUEMÁTICO GENERAL 

Sacar de EasyEda. 

IDENTIFICACIÓN DE PUNTOS HOMÓLOGOS UNA VEZ REALIZADOS 

Texto.

IMPLEMENTACIÓN

Se agregaron puntos de prueba en el circuito para realizar todas las mediciones necesarias. 

CONSUMO DE CORRIENTE DE COMPONENTES 

En base a la corriente medida en las resistencias de bajo valor y el display de las fuentes de tensión utilizadas. 

CÁLCULO DEL CICLO DE TRABAJO UTILIZADO 

D=TonPWM/Tswitching=125kHz*TonPWM

Destacar la estabilidad de la frecuencia de conmutación y la variación de la amplitud de la forma de onda PWM con la variación del ciclo de trabajo. 

POSIBILIDAD DE UTILIZAR DISIPADOR PARA MOSFET

PRIMERAS PRUEBAS

Al conectar la señal de salida PWM del TL494 al convertidor forward, debido a la carga vista, su amplitud disminuye de 12V a menos de 0.1V. 
Actualmente con la etapa de ganancia de corriente, la señal PWM disminuye a una amplitud de XV y presenta una leve deformación.

¿BUFFERS YA NO ESTÁN MÁS?
Prueba con LM324N
Producto Ganancia por Ancho de Banda: 1.3MHz y Slew Rate: 0.4V/us. 
Alimentación 12V. Los amplificadores operacionales que componen al circuito integrado no son rápidos ya que no pueden seguir a la tensión en su entrada. 
ADJUNTAR IMÁGEN!
Presenta tiempos de subida y de bajada muy lentos. En base a la medición realizada se necesita aproximadamente 1V/us de slew rate. 
Dado que el problema es el slew rate se busca un comparador en lugar de un amplificador operacional. 
DUDOSO:
La principal diferencia entre un amplificador operacional y un comparador es que el comparador
 no tiene el capacitor en su estructura interna que es el que limita el slew rate en el amplificador operacional. 

COMPARADOR LM339

Rango de alimentación: 2-36V
COMPLETAR
Velocidad: 1.3us

Se realizaron pruebas con un comparador. 
Salida por colector. Corriente de salida: 6-16mA. Necesita una resistencia de pull-up que fije esta corriente de $2k\Omega$ y $750\Omega$ respectivamente. 
Las señales de entrada eran la propia señal PWM que se desea mejorar y una referencia cuya amplitud 
sea la mitad de la primer señal implementada mediante un divisor de tensión. 
Adicional: Realimentación positiva con schmitt trigger para las transiciones. ¿Se implementó?+

NO ES NECESARIO LA ETAPA DE GANANCIA 2 Y BUFFER ¿Por qué ya no es necesaria?

Dado que la corriente máxima por colector del TL494 es 250mA, el integrado puede comandar directamente al driver del MOSFET high side. 

DRIVER

Los capacitores de desacople actúan como un cortocircuito en la frecuencia de operación. 
Diodo evita la parte negativa. 
Analizar la Vg para no superar la máxima. 
El transformador sin componentes adicionales consumía demasiada corriente porque se lo alimentaba con corriente continua. 