\section{Problemas afrontados y soluciones}

TL494

Una incorrecta elección de Rt y Ct fuera de los valores recomendados generaba variaciones muy grandes de la frecuencia de switching, 
lo cual causaba inestabilidad en la tensión de salida. 

MEDICIÓN DE INDUCTANCIA 

Con un medidor de inductancia se obtuvo el valor de la inductancia de magnetización del transformador.
Con los extremos de un bobinado a circuito abierto, se realizó la conexión del otro bobinado al instrumento y se obtuvo:
Transformador de señal E25:
Lm=0.186mH; 
Transformador E70
Lm=;    

OSCILACIONES

Las oscilaciones de alta frecuencia (superior a 125kHz) observadas en las formas de onda se pueden deber a cualquier tanque LC. 

COMPARADOR

Se realizaron pruebas con un comparador. 
Salida por colector. Corriente de salida: 6-16mA. Necesita una resistencia de pull-up que fije esta corriente de $2k\Omega$ y $750\Omega$ respectivamente. 
Las señales de entrada eran la propia señal PWM que se desea mejorar y una referencia cuya amplitud 
sea la mitad de la primer señal implementada mediante un divisor de tensión. 
Adicional: Realimentación positiva con schmitt trigger para las transiciones. ¿Se implementó?

CABLES

Para reducir el ruido se utilizó la menor longitud posible para los cables utilizados en las conexiones. 

ESQUEMÁTICO GENERAL 

Sacar de EasyEda. 

IDENTIFICACIÓN DE PUNTOS HOMÓLOGOS UNA VEZ REALIZADOS 

IMPLEMENTACIÓN

Se agregaron puntos de prueba en el circuito para realizar todas las mediciones necesarias. 

DRIVER

Falta completar la sección.  
