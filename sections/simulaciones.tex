\section{Simulaciones}

TL494

1) Señal PWM

Se toma la salida por el colector de los transistores del circuito integrado. 
Su forma de onda son pulsos rectangulares. 
Amplitud: 0V a Vdrv dados por la tensión de alimentación.
Frecuencia de 125kHz dada por el capacitor Ct=1nF y $Rt=8k\Omega$ mediante el potenciómetro. 
Tiempo de encendido: permite controlar el ciclo de trabajo mediante un potenciómetro. 
Esta señal se mide en 3 condiciones diferentes: 
A) Sin el convertidor forward conectado
B) Con el convertidor forward conectado pero sin su alimentación de 36V
C) Con el convertidor forward conectado y alimentado

2) Diente de Sierra 

Tensión en el capacitor Ct. 

3) Tensión en el puerto DTC 

Etapa de ganancia de corriente

1) Corriente sin la etapa

2) Corriente de entrada y de salida con la etapa 

3) Tensión a la salida sin la etapa

3) Tensión a la salida con la etapa

Driver

1) Tensiones en el transformador de señal 

Forma de onda: Rectangular 
Amplitud: -Vc a Vdrv-Vc 
Son iguales dada la relación 1:1 del transformador. 

2) Tensión Gate del MOSFET high side 

Forma de onda: Rectangular 
Amplitud: -Vd a Vdrv-Vd
Vd: Tensión en el diodo del secundario

3) Corriente de salida

Compuesta por:

A) Corriente magnetizante

Forma de onda: triangular
Se mide en la Rc del primario. 

B) Corriente por Rgs

Forma de onda: rectangular

C) Corriente por Gate

Forma de onda: diente de sierra invertida y espejada con tiempo muerto 

Convertidor Forward Doble Switch 

1) Vgs de ambos MOSFETs

2) Vds de ambos MOSFETs

3) Idrain de ambos MOSFETs

4) Tensiones en el E70

5) Corriente en el primario

Medir la componente continua. 

6) Caída de tensión en los diodos

7) Corriente en el inductor junto con su ripple

8) Corriente de salida junto con su ripple 

Idealmente medir 1A-1.6A del modo elegido. 

9) Tensión de salida junto con su ripple 