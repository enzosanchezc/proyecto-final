\section{Implementación}

Se diseñó un circuito impreso y se construyó un prototipo funcional para cumplir con las especificaciones propuestas. 
Debido a que el costo de una batería de litio de las características necesarias es elevado,
se redujo la tensión de salida del conversor a un límite de 12.6V,
tensión que normalmente se utiliza para recargar baterías de computadoras portátiles.

Debido a la extensión del proyecto no se llegó a diseñar el circuito en su totalidad,
por lo que solo se implementó el conversor de tensión continua
junto con el generador de señal PWM y el driver.

Las primeras pruebas se realizaron en una placa de cobre perforada. 
Para una mayor seguridad se optó por reemplazar la primera etapa de conversión AC-DC por una fuente regulable cuya tensión de entrada sea de 36V.
Esta fuente tiene una corriente máxima de 1 A y por lo tanto es capaz de proveer la corriente necesaria que consume el circuito, con ciertas limitaciones.

Quedó pendiente por implementar todo el circuito de control,
junto con las protecciones necesarias para las baterías.
El rectificador a la entrada del conversor tampoco se llegó a implementar.