\section{Diseño}

\subsection{Transformador}

La potencia aparente del transformador es la suma de la potencia de entrada y la potencia de salida:

$$ P_{t}=P_{i}+P_{o}=\frac{P_{o}}{\eta}+P_{o}=(1+\frac{1}{\eta})P_{o} $$

La tensión en el bobinado primario está dada por: 

$$ V_{1}=K_{t} f N_{1} \phi_{m} $$

donde $f$ es la frecuencia de conmutación de la señal de entrada y 
$K_t$ es un factor que vale 4.44 si la forma de onda es sinusoidal o 4 si es rectangular.

El flujo se relaciona con el área de la sección transversal de la trayectoria del flujo $A_{c}$ y la densidad de flujo $B$ de la siguiente manera:

$$ \Phi=BA $$

Densidad de corriente:

$$ J=K_{j} A_{p}^{x} $$

donde $K_{j}$ y $x$ son constantes que dependen del núcleo magnético dadas por la siguiente tabla:

INSERTAR TABLA!

El factor $A_{p}$ se puede calcular como: 

$$ A_{p}=\left[\frac{P_{t} \times 10^{4}}{K_{t} f B_{m} K_{u} K_{j}}\right]^{\frac{1}{1+x}} [{cm}^{4}] $$

donde Ku es el factor de relleno que varía entre 0.4 y 0.6 y Bm es la densidad de flujo máxima.

La cantidad de alambre de cobre y la cantidad de material del núcleo (por lo general ferrita de hierro) determinan la capacidad de potencia del transformador. 
Existen distintos tipos de núcleo como Toridal core, Pot core, Power core, E-laminated core, EI-core, C-core, Single-coil, Tape-wound core, etc. 
En este proyecto se trabajará con los núcleos disponibles provistos por la cátedra: E70, E30 y E25.  

Especificaciones obtenidas por simulación:
\begin{itemize}
    \item Frecuencia de la forma de onda rectangular: $f=125kHz$
    \item Tensión máxima aplicada a la bobina del primario: $V_{1_{max}}=350V$
    \item Tensión de salida en el secundario: $V_{0}=12.6V$
    \item Corriente de salida en el secundario: $I_{0}=1.6A$
    \item Eficiencia: $n=95\%$
\end{itemize}

INSERTAR IMÁGEN DE SIMULACIÓN V1MAX!

\paragraph{Cálculos}

Potencia de salida:

$$ P_{o}=V_{o}*I_{o}=12.6V*1.6A=20.16W $$

Potencia aparente: 

$$Pt=Po+Pi=Po*\left(1+\frac{1}{n}\right)=20.16W*\left(1+\frac{1}{0.95}\right)=41.38W  $$

INSERTAR IMÁGEN B-H DE KIKE!

En base a las curvas de magnetización se observa como las ferrites no soportan tanto flujo como el hierro y como disminuye la permeabilidad para campos mayores.  
Presenta el ciclo de histéresis y comienza a saturar en B=300mT. 
La saturación implica mayor circulación de corriente que conlleva a que el núcleo deje de responder. 
La saturación se da en campos menores si la temperatura aumenta con el funcionamiento del dispositivo. 
Por lo tanto se elige una inducción magnética máxima de $B_{m}=100mT$ para estar lejos del codo y asegurarnos tener la permeabilidad especificada en las hojas de datos. 

Para el núcleo E70, el área efectiva es: 

$$ A_{e}=279{mm}^{2} $$

Número de vueltas del primario:

$$ N_{1}=\frac{V_{1_{max}}}{K_{t}*f*B_{m}*A_{a}}=\frac{350V}{4*125KHz*100mT*0.000279}=25.09\simeq 25 $$

Número de vueltas del secundario: 

$$ N_{2}=N_{1}*\frac{V_{2}}{V_{1}} $$

Como no se requiere ni desea elevar o reducir la tensión del secundario, se elige una relación de vueltas 1:1. 
Por lo tanto, el número de vueltas del bobinado secundario es igual al del primario. 

$$ N_{2}=25 $$

$$ A_{p}=\left[\frac{41.38\times 10^{4}}{4*125kHz*100mT*0.4*366}\right]^{\frac{1}{1-0.14}} [cm^4] $$

$$ J=366\times X^{-0.14} [A/cm^2] $$

$$ J=584 $$

\paragraph{Saturación magnética}

Cualquier componente de continua puede causar la saturación magnética del núcleo, 
elevando la corriente magnetizante. 
Para minimizar los efectos de la saturación, se puede utilizar un núcleo más grande o incluir un entrehierro.
El entrehierro permite que exista en el núcleo, además de una zona con alta permeabilidad propia del material magnético, 
otra zona de baja permeabilidad propia del espacio de aire. 
De esta forma en condiciones normales el flujo circula por el material magnético y en caso de saturación circula por el entrehierro 

\subsection{Inductor}

Se utiliza para almacenar energía y permitir su transferencia. 
Por el inductor circula una corriente continua. Si la misma es muy elevada puede saturar el núcleo magnético. 

Requisitos de diseño: 

Inductancia: L=400uH
Corriente continua: Idc=1A

Procedimiento para el diseño de un inductor con entrehierro:

1) Calculo la energía que debe almacenar el inductor

E=0.5*L*Idc^2=200uHA^2

2) Elegir el tamaño del núcleo.

En base a la disponibilidad, se elige un E30 con material N27. 
Sus parámetros son:

Longitud efectiva: le=67mm
Área efectiva: Ae=60mm^2
Para gap=0mm:
Al=1800nH y ue=1600
Para gap=0.1mm:
Al=630nH y ue=562
Para gap=0.18mm:
Al=400nH y ue=353
Para gap=0.34mm:
Al=200nH y ue=179

3) Verificar si la energía que puede almacenar el inductor sin entrehierro es suficiente. 

Con Bm=100mT:
Esingap=(0.5*(Bm^2)*volumen)/(u0*ue)=(0.5*(Bm^2)*Ae*le)/(u0*ue)=10uHA^2
Con Bm=200mT:
Esingap=(0.5*(Bm^2)*volumen)/(u0*ue)=(0.5*(Bm^2)*Ae*le)/(u0*ue)=40uHA^2

En ambos casos es menor a la energía que debe almacenar el inductor por lo que se evidencia la necesidad de incluir un entrehierro. 

4) Dimensionar el entrehierro 𝑙0 para almacenar la energía 𝐸 con 𝐵máximo=100mT.
En caso de ser necesario puede llevarse a Bmáximo=200mT sin problema. 

Con Bm=100mT:
l0=(2*E*u0)/(Ae*Bm^2)=0.84mm
Con Bm=200mT:
l0=(2*E*u0)/(Ae*Bm^2)=0.2mm

Gap necesario con un núcleo tipo E:

Con Bm=100mT:
g=l0/2=0.42mm
Con Bm=200mT:
g=l0/2=0.1mm

5) Dimensionar el número de vueltas N para conseguir la inductancia L requerida.

Con Bm=100mT:
N=sqrt((L*l0)/(Ae*u0))=67 vueltas
Con Bm=200mT:
N=sqrt((L*l0)/(Ae*u0))=34 vueltas

6) Construir el inductor y medir L. Si no verifica el valor deseado corregir
considerando 𝐿 = 𝐴𝐿𝑁2 (el valor de 𝐴𝐿 que corresponde a 𝑙0 se obtiene con
el número de vueltas del primer intento).

Mediante la constante de tiempo en un circuito RL con una forma de onda cuadrada a la entrada se estima la inductancia en L=450uH. 

7) Verificar que el valor de L se mantiene con la corriente de DC de trabajo.
Esto generalmente se realiza sobre el propio convertidor midiendo la
tensión aplicada sobre el inductor y la pendiente de la corriente.