\section{Problemas afrontados}

Una incorrecta elección de $R_t$ y $C_t$ para el TL494 fuera de los valores recomendados por el fabricante generaba variaciones muy grandes de la frecuencia de switching, 
lo cual causaba inestabilidad en la tensión de salida. 

La frecuencia de conmutación ajustada mediante el potenciómetro posee una gran estabilidad.

La amplitud de la forma de onda PWM disminuye al aumentar el ciclo de trabajo con el convertidor conectado a la salida del TL494. 

Dado que la carga que genera el convertidor sobre el TL494 es muy alta, sin una etapa de ganancia de corriente, la forma de onda PWM 
se distorsionaba y disminuía notablemente su amplitud, lo que causaba que los MOSFETs no se saturaran y aumentaran demasiado su temperatura.
Los resultados obtenidos con la inclusión de esta etapa fueron una mejora en la forma de onda de la señal PWM y un incremento notorio en su amplitud.