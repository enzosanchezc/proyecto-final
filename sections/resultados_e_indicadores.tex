\section{Metas}

%La evaluación de un proyecto juzga el cumplimiento de los objetivos. 

%Las metas operacionalizan y cuantifican a los objetivos, convirtiéndolos en logros específicos.
%Determinan cuándo, dónde y cuánto se realizarán de los mismos. Los definen en términos medibles.

Se realizarán dos (2) entregas: un archivo de simulación a mediados de Julio,
cuando finalice la etapa de diseño,
y un prototipo de cargador funcional en el mes de Diciembre,
junto con el manual de instrucciones y el informe final.

Las metas del proyecto con sus respectivos indicadores son las siguientes:

%Los indicadores establecen los criterios de éxito del proyecto, es decir, especifican cómo se miden las metas planteadas. 
\begin{itemize}
    \item Desviaciones menores al 5\% de la tensión nominal y corriente nominal requeridas en el perfil de carga
    \item Eficiencia superior al 70\% % DUDOSO
    \item Correcto funcionamiento del LED para cada uno de los estados del cargador
    \item Desconexión de la alimentación en caso de cortocircuito
    %\item Tiempo máximo de carga?
    %\item Funcionamiento correcto en todo el rango de entrada y en ambas frecuencias. ???
\end{itemize}