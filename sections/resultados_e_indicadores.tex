\section{Metas}

La evaluación de un proyecto juzga el cumplimiento de los objetivos. 

Las metas operacionalizan y cuantifican a los objetivos, convirtiéndolos en logros específicos.
Determinan cuándo, dónde y cuánto se realizarán de los mismos. Los definen en términos medibles.





Los indicadores establecen los criterios de éxito del proyecto, es decir, especifican cómo se miden las metas planteadas. 
\item Desviaciones menores al 5\% de la tensión y corriente requeridas en el perfil de carga 
\item Eficiencia superior al 70\%
\item Correcto funcionamiento del LED para cada uno de los estados del cargador
\item Desconexión de la alimentación en caso de cortocircuito
\item %tiempo máximo de carga?
\item Funcionamiento correcto en todo el rango de entrada y en ambas frecuencias. 
Productos que el proyecto espera entregar:
Primera entrega de la simulación completa del circuito
Segunda entrega con el prototipo final que cumplirá con todas las especificaciones y su manual de uso