\section{PCB}

A continuación se detallan todos los criterios aplicados para cada componente en base a las recomendaciones 
proporcionadas por la cátedra, los integrantes del ATEI y las especificadas en las hojas de datos de los circuitos integrados.  

ORIFICIOS

Orificios de 0.7mm/0.8mm, independiente del componente. En caso de necesitar uno más grande se lo hace con mecha.
Por ejemplo, para jumpers, potenciómetros y conectores macho el orificio es de 1mm. 

PISTAS

Con corrientes más altas se necesitan pistas más anchas. 
Por ello, las pistas de potencia se hacen lo más cortas, directas y gruesas posibles mediante la siguiente regla: 0.381 mm x Ampere.
Pistas finas de 1mm y pistas gruesas de 2mm. 
La separación entre las pistas y el plano de tierra o cualquier otra pista es de 0,4 mm/0.5mm. 
Se evitó que las pistas tengan ángulos de 90°. 
Si en el planchado se trabaja con una temperatura excesiva las pistas se ensanchan. 
Para evitar contactos entre pines, las pistas ingresan a los circuitos integrados de forma paralela. 

PADS

Se les dio forma de ... COMPLETAR!
Los pads de al menos 2mm.
En los casos donde las pistas están muy cerca de los pads, se achican un poco a costa de perder área para soldadura.  

TIERRA 

Con el fin de mantener la aislación provista por el transformador principal, 
el plano de tierra se dividió en dos: una parte para el lado primario y la otra para su secundario. 
Su forma es cuadrada para que visualmente sean fáciles de encontrar.

TORNILLOS

En cada extremo de la placa se realizaron orificios para los tornillos de 6mm de diámetro,
 que es la medida de la cabeza de tornillo típica. 

PUNTOS DE PRUEBA

Se colocaron puntos se prueba para poder conectar los dos transformadores a la placa. 

RESISTENCIAS DE BAJO VALOR 

Se colocaron resistencias de bajo valor en serie con cada componente o lugar
 en el que se desea medir corriente para contrastar con los resultados obtenidos por simulación. 

OTROS COMPONENTES 

Resistencias 2.5mm
Puentes de 3mm
Led de 1.8mm 
Mosfet y resistencia de potencia de 3.5mm

Se comprobó que todos los componente elegidos en el software tengan la misma distribución de pines que en su versión física. 
Se imprimió una primera versión y se realizaron correcciones en base a los resultados obtenidos al probar los componentes. 